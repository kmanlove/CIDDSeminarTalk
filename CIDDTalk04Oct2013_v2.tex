\documentclass[fleqn,xcolor=table]{beamer}
%\usetheme{KEZMarburgThreeColor}
\usetheme{Madrid}
\definecolor{kezgreen}{HTML}{459C57}
\definecolor{grey}{HTML}{CDC9C9}
\definecolor{navy}{RGB}{0, 0, 128}
\definecolor{darkred}{RGB}{139, 0, 0}
\usepackage{pgf,pgfnodes, pgfpages}
\usepackage[utf8]{inputenc}
\usepackage[T1]{fontenc}
\usepackage{transparent}
\usepackage{graphicx}
\usepackage[absolute,overlay]{textpos}
\newcommand{\semitransp}[2][20]{\color{fg!#1}#2}
%\und{\semitransp}[2][35]{\color{fg!#1}#2}secolortheme{orchid}
\usecolortheme{dove}
\setbeamertemplate{blocks}[rounded][shadow=TRUE]
%\newcommand\T{\rule{0pt}{3.6ex}}
%\newcommand\B{\rule[-1.2ex]{0pt}{0pt}}
\usepackage{color}
%\setbeameroption{show notes} %un-comment to see the notes
%\setbeameroption{show notes on second screen=right}




%\tletter
\setbeamertemplate{footline}
{
  \leavevmode%
    \hbox{%
      \begin{beamercolorbox}[wd=.333333\paperwidth,ht=2.25ex,dp=1ex,center]{author
      in head/foot}%
          \usebeamerfont{author in
	  head/foot}\insertshortauthor%~~\beamer@ifempty{\insertshortinstitute}{}{(\insertshortinstitute)}
	    \end{beamercolorbox}%
	      \begin{beamercolorbox}[wd=.333333\paperwidth,ht=2.25ex,dp=1ex,center]{title
      in head/foot}%
          \usebeamerfont{title in head/foot}\insertshorttitle
	    \end{beamercolorbox}%
	      \begin{beamercolorbox}[wd=.333333\paperwidth,ht=2.25ex,dp=1ex,right]{date
      in head/foot}%
          \usebeamerfont{date in head/foot}\insertshortdate{}\hspace*{2em}
	      \insertframenumber{} / \inserttotalframenumber\hspace*{2ex} 
	        \end{beamercolorbox}}%
	  \vskip0pt%
  }
  \makeatother
%setbeameroption{show notes on second screen=right}
%\institution{Penn State}
\author{Kezia Manlove}
\title{Social connectivity/Offspring mortality}
\begin{document}

\begin{frame}[t]

\vspace{.05in}
%
%\transparent{1}
%\parbox[t]{3.5in}{\center{\color{darkblue}{Linking Adult Social Connectivity Patterns to Offspring Mortality in Bighorn Sheep}}
%\vspace{.15in}
%
%\parbox[c]{3.5in}{\center{\color{kezred}{Big problems, small data}}}

{\center{\color{navy}{Linking Adult Social Connectivity Patterns to
Offspring Mortality in Bighorn Sheep}}}
\vspace{.15in}

%
\includegraphics[width = \textwidth]{HornsInGrass_Cropped.JPG}
\vspace{.2in}

\footnotesize{\color{navy}{Kezia Manlove}}

\vspace{.05in}

{\color{navy} \today}
\vspace{.2in}
%}

\note{\begin{itemize}
	\item Collaborative effort: ID, WA, OR; PSU, WSU, USGS.
	\item Second year grad student.
	\item Fourth year with project.  
	\item Pete is advisor
	\item \textbf{Interest}: develop methods that rely on available data to
		understand diseases in wildlife
	\end{itemize}
	}

\end{frame}

{\usebackgroundtemplate{\transparent{.35}\parbox[c][\paperheight][c]{\paperwidth}{\centering\includegraphics[width=\paperwidth]{CoverRam_Kamloops.jpg}}}

	\section{Introduction}
\begin{frame}[t]
	\frametitle{\color{darkred}Pneumonia in Bighorn sheep}
\begin{itemize}
\item \small {\color{navy}Limits bighorn sheep population growth/reintroduction efforts in
	US Rocky Mountains }
\item \small {\color{navy}Historically linked to nearby domestic sheep
}
\item \small {\color{navy} Reservoir removal hasn't solved the problem}

\end{itemize}

{\color{darkred} Stereotypical Outbreak}
\newline
%
	\includegraphics[height = .5\textheight]{PneumoniaOutbreakCartoon.pdf}
%\begin{itemize}
%{\small
%\item {\color{navy} Reservoir removal hasn't mitigated pneumonia impacts}
%%\item {\color{keztextblue}Reservoir removal hasn't mitigated pneumonia impacts}
%%	\item \textit{Controlled data are sparse; basic knowledge is based on
%%field observations}
%
%}
%\end{itemize}

\note{
	\begin{itemize}
		\item Bighorn sheep in trouble
		\item Disease is reason
	\end{itemize}
}

\end{frame}
}
%
%\begin{frame}	
%\frametitle{A stereotypcial pneumonia epidemic}
%	\includegraphics[width = \textwidth]{PneumoniaOutbreakCartoon.pdf}
%
%\end{frame}

\begin{frame}
	\frametitle{\color{darkred} Heterogeneity in disease impacts}
	\begin{columns}[t]
		\column{.65\textwidth}
		\vspace{.05in}

%		\begin{minipage}[c][.6\textheight][c]{\linewidth}
	\includegraphics[width =\textwidth]{Cassirer2013_LambMortHazFig_V2.pdf}
%	\hspace{.05in}
%	\includegraphics[width =
%	.45\textwidth]{JAEPaper_LambHazards_11Feb2013.pdf}
%		\end{minipage}
	\newline
	{\footnotesize \color{white}{Hypothesis: Variation explained by
	incomplete exposure}}
		\column{.35\textwidth}
			\begin{minipage}[t][.8\textheight][t]{\linewidth}
				\vspace{.4in}
%
			\begin{itemize}
				{\footnotesize
				\item[] {\color{white} \scriptsize Pneumonia $\implies$ diminished recruitment}
				\item[] {\color{white} \scriptsize Magnitude of declines varies} 
				}
			\end{itemize}
		\end{minipage}
	\end{columns}
		
\end{frame}

\begin{frame}
	\frametitle{\color{darkred} Heterogeneity in disease impacts}
	\begin{columns}[t]
		\column{.65\textwidth}
		\vspace{.05in}

%		\begin{minipage}[c][.6\textheight][c]{\linewidth}
	\includegraphics[width =\textwidth]{Cassirer2013_LambMortHazFig_V2.pdf}
	\newline
	{\footnotesize \color{white}{Hypothesis: Variation explained by
	incomplete exposure}}
%	\hspace{.05in}
%	\includegraphics[width =
%	.45\textwidth]{JAEPaper_LambHazards_11Feb2013.pdf}
%		\end{minipage}
		\column{.35\textwidth}
			\begin{minipage}[t][.8\textheight][t]{\linewidth}
				\vspace{.4in}

			\begin{itemize}
				{\footnotesize
				\item {\color{navy} \scriptsize Pneumonia
					reduces summer lamb survival}
					\vspace{.1in}
				\item {\color{navy} \scriptsize Impact is
					variable in magnitude} 
				}
			\end{itemize}
		\end{minipage}
%		\end{column}
	\end{columns}
		
\end{frame}

\begin{frame}
	\frametitle{\color{darkred} Heterogeneity in disease impacts}
	\begin{columns}[t]
		\column{.65\textwidth}
		\vspace{.05in}

%		\begin{minipage}[c][.6\textheight][c]{\linewidth}
	\includegraphics[width =\textwidth]{Cassirer2013_LambMortHazFig_V2.pdf}
	\newline
	\vspace{.1in}

	{\footnotesize \color{darkred}{Hypothesis: Variation explained by
	incomplete exposure}}
%	\hspace{.05in}
%	\includegraphics[width =
%	.45\textwidth]{JAEPaper_LambHazards_11Feb2013.pdf}
%		\end{minipage}
		\column{.35\textwidth}
			\begin{minipage}[t][.8\textheight][t]{\linewidth}
				\vspace{.4in}

			\begin{itemize}
				{\footnotesize
				\item {\color{navy} \scriptsize Pneumonia
					reduces summer lamb survival}
				\vspace{.1in}
				\item {\color{navy} \scriptsize Impact is
					variable in magnitude} 
				}
			\end{itemize}
		\end{minipage}
%		\end{column}
	\end{columns}
\end{frame}
%
%\section{Epidemic Scale}
%\begin{frame}[t]
%\frametitle{\color{darkred} Is an entire population vulnerable to disease?}
%\vspace{.1in}
%
%\begin{itemize}
%	\item[] {\color{darkred} Relevance}
%		\begin{itemize}
%		\item \color{navy} \footnotesize Management actions occur at population level
%			\vspace{.1in}
%			
%			\item \color{navy} \footnotesize Anecdotally, mortalities cluster locally within populations
%		\end{itemize}
%		\vspace{.2in}
%
%	\item[] {\color{darkred} Specific questions}
%		\begin{itemize}
%			\item \color{navy} \footnotesize Do populations differ systematically in their
%				social contact structures?
%%				\begin{itemize}
%%					\item {\color{navy} Hypothesis:
%%						Populations will differ in
%%					their contact patterns.}
%%				\end{itemize}
%				\vspace{.1in}
%
%			\item \color{navy} \footnotesize What does clustering of lamb mortalities mean re: adult
%			shedding rates?
%%				\begin{itemize}
%%					\item {\color{keztextblue} Hypothesis:
%%						Lamb mortality patterns will be
%%					consistent with population-level social
%%				contact patterns.}
%%				\end{itemize}
%		\end{itemize}
%	\end{itemize}
%\end{frame}
%%
{\usebackgroundtemplate{\transparent{.35}\parbox[c][\paperheight][c]{\paperwidth}{\centering\includegraphics[height=\paperheight]{CoverEwe.jpg}}}
\begin{frame}[t]
	\begin{itemize}
		\item[] \color{navy} \footnotesize \textbf{Do bighorn sheep populations vary in their social
			structuring?}
			\vspace{.8in}
		\item[] \color{navy} \footnotesize \textbf{Does social
				structuring account for variation in
			summer lamb survival?}
			\vspace{.8in}
		\item[] \color{navy} \footnotesize \textbf{Can we use information on social
			groups to understand 
			epidemiological rates?}
	\end{itemize}
\end{frame}
}
%
%\section{Data Analysis}
%\begin{frame}[t]
%	\frametitle{\color{darkred} Data Collection}
%%\begin{frame}
%	\hspace{.18in}
%	\includegraphics[height = .48\textheight]{HelicopterInboundGIMP.jpg}
%	\hspace{.09in}
%	\includegraphics[height = .48\textheight]{1389WithLambGIMP.jpg}\\
%	\hspace{.18in}
%	\includegraphics[height = .42\textheight]{LostineSickLamb.jpg}
%	\hspace{.1in}
%	\includegraphics[height = .42\textheight]{DeadLambGIMP.jpg} 
%\end{frame}
%
%{\usebackgroundtemplate{\transparent{.35}\parbox[c][\paperheight][c]{\paperwidth}{\centering\includegraphics[width=\paperwidth]{CoverRam_Kamloops.jpg}}}


\begin{frame}[t]
\frametitle{\color{darkred} Social structure through contact networks}
%	\begin{itemize}
%		\item {\color{keztextblue}Network Construction}
	\begin{itemize}
		{\small
		\item \color{navy} Relocation data at population-summer scale 
			\begin{itemize}
				\item[] \color{navy} {\footnotesize 3 populations, 15
					years/population}
			\end{itemize}
		\item \color{navy} Built social contact networks of radiocollared ewes
%		\item Focussed on edgeweights and communities within networks
		}
	\end{itemize}
	\vspace{.3in}
%\end{itemize}
%	\includegraphics[width = .45\textwidth]{Imnaha2002.jpg}
%	\hspace{.05in}
%%	\includegraphics[width = .45\textwidth]{Redbird1998.jpg}
%\end{frame}
%
%\begin{frame}
%\frametitle{Networks}
%	\includegraphics[width = .45\textwidth]{Redbird1998.jpg}
%	\includegraphics[width = .45\textwidth]{Redbird1998_Revised.pdf}
%	\textwidth]{KeyPopComponentDescriptions_03Oct2013.pdf}
	\hspace{.15in}
%	\includegraphics[width = .45\textwidth]{Redbird1998.jpg}
%	\includegraphics[width = .45\textwidth]{Imnaha2002.jpg}
%	\includegraphics[width = .45\textwidth]{Imnaha2002_revised.pdf}
\note{
	\begin{itemize}
		\item Nodes = radiocollared ewes
		\item Edges link ewes observed together at least once
		\item edges weighted by association index
		\item Red = lamb died; blue = lamb survived through Oct 1
	\end{itemize}
}
\end{frame}


\begin{frame}[t]
\frametitle{\color{darkred} Social structure through contact networks}
%	\begin{itemize}
%		\item {\color{keztextblue}Network Construction}
	\begin{itemize}
		{\small
		\item \color{navy} Relocation data at population-summer scale 
			\begin{itemize}
				\item[] \color{navy} {\footnotesize 3 populations, 15
					years/population}
			\end{itemize}
		\item \color{navy} Built social contact networks of radiocollared ewes
%		\item Focussed on edgeweights and communities within networks
		}
	\end{itemize}
	\vspace{.3in}
%\end{itemize}
%	\includegraphics[width = .45\textwidth]{Imnaha2002.jpg}
%	\hspace{.05in}
%%	\includegraphics[width = .45\textwidth]{Redbird1998.jpg}
%\end{frame}
%
%\begin{frame}
%\frametitle{Networks}
%	\includegraphics[width = .45\textwidth]{Redbird1998.jpg}
	\includegraphics[width = .45\textwidth]{Redbird1998_Revised.pdf}
%	\textwidth]{KeyPopComponentDescriptions_03Oct2013.pdf}
	\hspace{.15in}
%	\includegraphics[width = .45\textwidth]{Redbird1998.jpg}
%	\includegraphics[width = .45\textwidth]{Imnaha2002.jpg}
%	\includegraphics[width = .45\textwidth]{Imnaha2002_revised.pdf}
\note{
	\begin{itemize}
		\item Nodes = radiocollared ewes
		\item Edges link ewes observed together at least once
		\item edges weighted by association index
		\item Red = lamb died; blue = lamb survived through Oct 1
	\end{itemize}
}
\end{frame}

\begin{frame}[t]
\frametitle{\color{darkred} Social structure through contact networks}
%	\begin{itemize}
%		\item {\color{keztextblue}Network Construction}
	\begin{itemize}
		{\small
		\item \color{navy} Relocation data at population-summer scale 
			\begin{itemize}
				\item[] \color{navy} {\footnotesize 3 populations, 15
					years/population}
			\end{itemize}
		\item \color{navy} Built social contact networks of radiocollared ewes
%		\item Focussed on edgeweights and communities within networks
		}
	\end{itemize}
	\vspace{.3in}
%\end{itemize}
%	\includegraphics[width = .45\textwidth]{Imnaha2002.jpg}
%	\hspace{.05in}
%%	\includegraphics[width = .45\textwidth]{Redbird1998.jpg}
%\end{frame}
%
%\begin{frame}
%\frametitle{Networks}
%	\includegraphics[width = .45\textwidth]{Redbird1998.jpg}
	\includegraphics[width = .45\textwidth]{Redbird1998_Revised.pdf}
%	\textwidth]{KeyPopComponentDescriptions_03Oct2013.pdf}
	\hspace{.15in}
%	\includegraphics[width = .45\textwidth]{Redbird1998.jpg}
%	\includegraphics[width = .45\textwidth]{Imnaha2002.jpg}
	\includegraphics[width = .45\textwidth]{Imnaha2002_revised.pdf}
\note{
	\begin{itemize}
		\item Nodes = radiocollared ewes
		\item Edges link ewes observed together at least once
		\item edges weighted by association index
		\item Red = lamb died; blue = lamb survived through Oct 1
	\end{itemize}
}
\end{frame}

\begin{frame}[t]
\frametitle{\color{darkred} Do populations differ systematically in their social contact
patterns?}
\vspace{.2in}

	\includegraphics[width =
	\textwidth]{KeyPopComponentDescriptions_03Oct2013.pdf}
\note{
	\begin{itemize}
		\item Explain Imnaha on LHS
		\item ``just a few examples...''
	\end{itemize}
}

\end{frame}


{\usebackgroundtemplate{\transparent{.35}\parbox[c][\paperheight][c]{\paperwidth}{\centering\includegraphics[height=\paperheight]{CoverEwe.jpg}}}
\begin{frame}[t]
	\begin{itemize}
		\item[] \color{navy} \footnotesize \textbf{Do bighorn sheep populations vary in their social
			structuring?}
			\vspace{.1in}
			\color{darkred} \textbf{Yes.}

			\includegraphics[width =
	.3\textwidth]{KeyCompos_simple.pdf}
			\vspace{.1in}
		\item[] \color{navy} \footnotesize Does social structuring account for variation in
			summer lamb survival?
			\vspace{.8in}
		\item[] \color{navy} \footnotesize Can we leverage information on social groups to 
			understand epidemiological rates?
	\end{itemize}
\end{frame}
}
%{\setbeamercolor{background canvas}{bg = white}
%\begin{frame}
%	\frametitle{\color{darkred} Lamb survival models}
%%\begin{itemize}
%%\item {\color{keztextblue} Lamb Mortality Analysis} 
%			\begin{itemize}
%				{\small
%				\item Fit Cox proportional hazards models for
%					lamb survival
%				\item Estimated random effects at three levels
%					\begin{itemize}
%						\item Component within
%							population-year
%						\item Population-year
%						\item Population
%					\end{itemize}
%				\item Compared variance captured at each level
%					of structuring
%				}
%			\end{itemize}
%%	\end{itemize}
%
%\end{frame}
%}


\begin{frame}[t]
	\frametitle{\color{darkred} Lamb Survival Models}
	
	\vspace{.2in}
	\begin{itemize}
		\item {\color{navy} Examined lamb survival in pneumonia and
			healthy years}
		\item {\color{navy} Focused on nested shared frailty structure}
			\begin{itemize}
				\item[] \color{navy} Populations
				\item[] \color{navy} Years in populations
				\item[] \color{navy} Groups in years in populations
			\end{itemize}
	\end{itemize}
	\vspace{.2in}

%	\footnotesize{\color{navy} For lambs in the \color{darkred}$k^{th}$
%		component \color{navy}of the
%		\color{darkred}$j^{th}$ year \color{navy}in the
%		\color{darkred}$i^{th}$ population\color{navy}, hazard at time
%		$t$ ($h_{ijk}(t)$) is:
%\[h_{ijkl}(t)=h_{0}(t)exp\left[\mathbf{Z_{i}}b_{i} +\mathbf{Z_{ij}} b_{ij} +
%\mathbf{Z_{ijk}} b_{ijk}\right]\]
%}
	\note{``Shared frailty'' is survival modeling analog of REs}
\end{frame}

\begin{frame}[t]
	\frametitle{\color{darkred} Does social structuring account for
	variation in summer lamb survival?}
	\vspace{.2in}

	\includegraphics[width =
	\textwidth]{VarDecompExplanation_03Oct2013_slide1.pdf}
%\begin{itemize}
%	\item Data from 50 population-summers 
%		\begin{itemize}
%			\item Up to 15 years
%			\item Four populations
%		\end{itemize}
%%		\vspace{.2in}
%	
%\end{itemize}
%
%\includegraphics[height =
%.6\textwidth]{VarianceDecompDensities_01Oct2013V2.pdf}
\note{
	\begin{itemize}
		\item Scenario 1: all variation between populations $\implies$
			high variation in Pop effect
		\item Scenario 2: high variation within pops, low between pops
			$\implies$ high variation in group
		\item Scenario 3: low between pop variation
	\end{itemize}
}
\end{frame}

\begin{frame}[t]
	\frametitle{\color{darkred} Does social structuring account for
	variation in summer lamb survival?}
	\vspace{.2in}

	\includegraphics[width =
	\textwidth]{VarDecompExplanation_03Oct2013_slide2.pdf}
\end{frame}


{\usebackgroundtemplate{\transparent{.35}\parbox[c][\paperheight][c]{\paperwidth}{\centering\includegraphics[height=\paperheight]{CoverEwe.jpg}}}
\begin{frame}[t]
	\begin{itemize}
		\item[] \footnotesize \color{navy} Do bighorn sheep populations vary in their social
			structuring?
			\color{darkred} Yes.

			\includegraphics[width =
	.3\textwidth]{KeyCompos_simple.pdf}
			\vspace{.1in}
		\item[] \footnotesize \color{navy} \textbf{Does social structuring account for variation in
			summer lamb survival?}
			\color{darkred} \textbf{Yes.}

			\includegraphics[width =
	.3\textwidth]{REhists_simple.pdf}
			\vspace{.1in}
		\item[] \footnotesize \color{navy} Can we leverage information on social groups to 
			understand epidemiological rates?
	\end{itemize}
\end{frame}
}
%
%\begin{frame}[t]
%	\frametitle{\color{darkred} Summary thus far}
%	\vspace{.2in}
%
%	\begin{enumerate}
%		\item {\color{navy} Social substructuring differs between 
%			populations}
%		\vspace{.1in}
%
%		\item \color{navy} Social substructuring explains significant 
%			portion of variation in lamb survival
%	\end{enumerate}
%\vspace{.2in}
%\begin{center}
%	{\Large {\color{darkred} Can we use differences in social substructuring
%	to learn something about epidemiological rates?}}
%\end{center}
%\end{frame}
%
\section{Simulations}
\begin{frame}[t]
	\frametitle{\color{darkred} Simulation}
\footnotesize \color{navy}Estimate
\begin{enumerate}
	\item{\footnotesize \color{navy}Proportion of ewes transmitting to
		lambs ($\pi$)}
	\item{\footnotesize \color{navy}Disease-induced mortality rate in lambs
		($\alpha$)} 
\end{enumerate}

\vspace{.2in}
{\color{navy} Approach}
{\footnotesize
	\begin{itemize}
		\item \scriptsize \color{navy}Individual-based 
			\color{red}S\color{navy}usceptible-\color{red}I\color{navy}nfected structure
			\[I_{k, t} = \left\{\text{Infected neighbors
			of lamb }k \text{ at beginning of timestep }
	t\right\}\] \[P(k \text{ becomes infected in timestep t}) = 1-exp\left(\sum_{i \in
			I_{k, t}}Edgeweight_i\right)\]
		\item \scriptsize \color{navy}Simulated epidemics on empirical contact network
		\item \scriptsize \color{navy}Systematically varied $\pi$ and $\alpha$
		\item \scriptsize \color{navy}Compared simulated lamb mortality
			to observed lamb mortality
%			\begin{itemize}
%				\item \scriptsize \color{navy}Proportion of ewes that transmit
%					pneumonia to their lambs
%				\item \scriptsize \color{navy}Lamb mortality rate | Exposure
%			\end{itemize}
%		
%		\item Ran 100 replicate sims at 10 levels across each dimension
%			of parameter space
%		\item Sims ``successful'' if simmed lamb
%			mortality rate fell within 1 sd of
%			observed rate
%		\item ID'd regions of parameter space where sims were
%			successful at least 75\% of the time
	\end{itemize}
}
\end{frame}

\begin{frame}
\frametitle{\color{darkred} Can we leverage information on social groups to
udnerstand epidemiological rates?}
\vspace{.1in}

%	\includegraphics[width = .35\textwidth]{BlackButte1997.jpg}
%	\hspace{.05in}
%	\includegraphics[width = .55\textwidth]{BLACK_BUTTE_1997_Heatmap.pdf}
\begin{center}
\includegraphics[width =
	\textwidth]{BB97NetworkAndHeatmap_03Oct2013_slide1.pdf}
\end{center}

\end{frame}


\begin{frame}
\frametitle{\color{darkred} Can we leverage information on social groups to
understand epidemiological rates?}
\vspace{.1in}

%	\includegraphics[width = .35\textwidth]{BlackButte1997.jpg}
%	\hspace{.05in}
%	\includegraphics[width = .55\textwidth]{BLACK_BUTTE_1997_Heatmap.pdf}
\begin{center}
\includegraphics[width =
	\textwidth]{BB97NetworkAndHeatmap_03Oct2013_slide2.pdf}
\end{center}

\end{frame}


\begin{frame}
\frametitle{\color{darkred} Can we leverage information on social groups to
understand epidemiological rates?}
\vspace{.1in}

%	\includegraphics[width = .35\textwidth]{BlackButte1997.jpg}
%	\hspace{.05in}
%	\includegraphics[width = .55\textwidth]{BLACK_BUTTE_1997_Heatmap.pdf}
\begin{center}
\includegraphics[width =
	\textwidth]{BB97NetworkAndHeatmap_03Oct2013_slide3.pdf}
\end{center}

\end{frame}

\begin{frame}
\frametitle{\color{darkred} Can we leverage information on social groups to
understand epidemiological rates?}
\vspace{.1in}

%	\includegraphics[width = .35\textwidth]{BlackButte1997.jpg}
%	\hspace{.05in}
%	\includegraphics[width = .55\textwidth]{BLACK_BUTTE_1997_Heatmap.pdf}
\begin{center}
\includegraphics[width =
	\textwidth]{BB97NetworkAndHeatmap_03Oct2013.pdf}
\end{center}

\end{frame}
%\begin{frame}
%\begin{frame}
%\frametitle{Broadscale Simulation Outputs}
%
%\begin{itemize}
%	\includegraphics[width = .75\textwidth]{ParamCoverage_inkscape.pdf}
%\end{itemize}
%\end{frame}
%

%{\usebackgroundtemplate{\transparent{.35}\parbox[c][\paperheight][c]{\paperwidth}{\centering\includegraphics[height=\paperheight]{CoverEwe.jpg}}}
%\begin{frame}[t]
%	\begin{itemize}
%		\item[] \color{navy} \footnotesize Do bighorn sheep populations vary in their social
%			structuring?
%			\color{darkred} Yes.
%
%			\includegraphics[width =
%	.3\textwidth]{KeyCompos_simple.pdf}
%			\vspace{.1in}
%		\item[] \color{navy} \footnotesize Does social structuring account for variation in
%			summer lamb survival?
%			\color{darkred} Yes.
%
%			\includegraphics[width =
%	.3\textwidth]{REhists_simple.pdf}
%			\vspace{.1in}
%		\item[] \color{navy} \footnotesize \textbf{Can we leverage
%				information on social groups to understand 
%			epidemiological rates?}
%			\color{darkred} \textbf{Yes.}
%
%			\includegraphics[width =
%	.2\textwidth]{Heatmap_simple.pdf}
%	\end{itemize}
%\end{frame}
%}
%
{\usebackgroundtemplate{\transparent{.35}\parbox[c][\paperheight][c]{\paperwidth}{\centering\includegraphics[height=\paperheight]{CoverEwe.jpg}}}
\begin{frame}[t]
	\begin{itemize}
		\item[] \color{navy} \footnotesize Do bighorn sheep populations vary in their social
			structuring?
			\color{darkred} Yes.
	\begin{columns}
		\column{.3\textwidth}
			\includegraphics[width =
			\textwidth]{KeyCompos_simple.pdf}
			\vspace{.1in}
		\column{.7\textwidth}
%			\footnotesize \color{darkred} Appropriate management
%			strategy may vary from population to population
	\end{columns}
		\item[] \color{navy} \footnotesize Does social structuring account for variation in
			summer lamb survival?
			\color{darkred} Yes.
	\begin{columns}
		\column{.3\textwidth}
			\includegraphics[width =
			\textwidth]{REhists_simple.pdf}
			\vspace{.1in}
		\column{.7\textwidth}
%			\footnotesize \color{darkred}Management efforts should
%			account for local structuring
	\end{columns}

			\vspace{.1in}
		\item[] \color{navy} \footnotesize \textbf{Can we leverage
				information on social groups to understand 
			epidemiological rates?}
			\color{darkred} \textbf{Yes.}

	\begin{columns}
		\column{.3\textwidth}
			\includegraphics[width =
			.8\textwidth]{Heatmap_simple.pdf}
			\vspace{.1in}
		\column{.7\textwidth}
%			\footnotesize \color{darkred}Don't overlook
%			heterogeneity in lamb outcomes
	\end{columns}
			\includegraphics[width =
	.2\textwidth]{Heatmap_simple.pdf}
	\end{itemize}
\end{frame}
}

\section{Conclusions}
{\usebackgroundtemplate{\transparent{.35}\parbox[c][\paperheight][c]{\paperwidth}{\centering\includegraphics[height=\paperheight]{CoverEwe.jpg}}}
\begin{frame}[t]
	\begin{itemize}
		\item[] \color{navy} \footnotesize Do bighorn sheep populations vary in their social
			structuring?
			\color{darkred} Yes.
	\begin{columns}
		\column{.3\textwidth}
			\includegraphics[width =
			\textwidth]{KeyCompos_simple.pdf}
			\vspace{.1in}
		\column{.7\textwidth}
%			\footnotesize \color{darkred} Appropriate management
%			strategy may vary from population to population
	\end{columns}
		\item[] \color{navy} \footnotesize Does social structuring account for variation in
			summer lamb survival?
			\color{darkred} Yes.
	\begin{columns}
		\column{.3\textwidth}
			\includegraphics[width =
			\textwidth]{REhists_simple.pdf}
			\vspace{.1in}
		\column{.7\textwidth}
%			\footnotesize \color{darkred}Management efforts should
%			account for local structuring
	\end{columns}

			\vspace{.1in}
		\item[] \color{navy} \footnotesize Can we leverage
				information on social groups to understand 
			epidemiological rates?
			\color{darkred} Yes.

	\begin{columns}
		\column{.3\textwidth}
			\includegraphics[width =
			.8\textwidth]{Heatmap_simple.pdf}
			\vspace{.1in}
		\column{.7\textwidth}
%			\footnotesize \color{darkred}Don't overlook
%			heterogeneity in lamb outcomes
	\end{columns}
			\includegraphics[width =
	.2\textwidth]{Heatmap_simple.pdf}
	\end{itemize}
\end{frame}
}
{\usebackgroundtemplate{\transparent{.35}\parbox[c][\paperheight][c]{\paperwidth}{\centering\includegraphics[height=\paperheight]{CoverEwe.jpg}}}
\begin{frame}[t]
	\begin{itemize}
		\item[] \color{navy} \footnotesize Do bighorn sheep populations vary in their social
			structuring?
			\color{darkred} Yes.
	\begin{columns}
		\column{.3\textwidth}
			\includegraphics[width =
			\textwidth]{KeyCompos_simple.pdf}
			\vspace{.1in}
		\column{.7\textwidth}
			\footnotesize \color{darkred} \textbf{Appropriate management
			strategy may vary from population to population}
	\end{columns}
		\item[] \color{navy} \footnotesize Does social structuring account for variation in
			summer lamb survival?
			\color{darkred} Yes.
	\begin{columns}
		\column{.3\textwidth}
			\includegraphics[width =
			\textwidth]{REhists_simple.pdf}
			\vspace{.1in}
		\column{.7\textwidth}
%			\footnotesize \color{darkred}Management efforts should
%			account for local structuring
	\end{columns}

			\vspace{.1in}
		\item[] \color{navy} \footnotesize Can we leverage
				information on social groups to understand 
			epidemiological rates?
			\color{darkred} Yes.

	\begin{columns}
		\column{.3\textwidth}
			\includegraphics[width =
			.8\textwidth]{Heatmap_simple.pdf}
			\vspace{.1in}
		\column{.7\textwidth}
%			\footnotesize \color{darkred}Don't overlook
%			heterogeneity in lamb outcomes
	\end{columns}
			\includegraphics[width =
	.2\textwidth]{Heatmap_simple.pdf}
	\end{itemize}
\end{frame}
}
%
%\begin{frame}[t]

{\usebackgroundtemplate{\transparent{.35}\parbox[c][\paperheight][c]{\paperwidth}{\centering\includegraphics[height=\paperheight]{CoverEwe.jpg}}}
\begin{frame}[t]
	\begin{itemize}
		\item[] \color{navy} \footnotesize Do bighorn sheep populations vary in their social
			structuring?
			\color{darkred} Yes.
	\begin{columns}
		\column{.3\textwidth}
			\includegraphics[width =
			\textwidth]{KeyCompos_simple.pdf}
			\vspace{.1in}
		\column{.7\textwidth}
			\footnotesize \color{darkred} Appropriate management
			strategy may vary from population to population
	\end{columns}
		\item[] \color{navy} \footnotesize Does social structuring account for variation in
			summer lamb survival?
			\color{darkred} Yes.
	\begin{columns}
		\column{.3\textwidth}
			\includegraphics[width =
			\textwidth]{REhists_simple.pdf}
			\vspace{.1in}
		\column{.7\textwidth}
			\footnotesize \color{darkred} \textbf{Management efforts should
			account for local structuring}
	\end{columns}

			\vspace{.1in}
		\item[] \color{navy} \footnotesize Can we leverage
				information on social groups to understand 
			epidemiological rates?
			\color{darkred} Yes.

	\begin{columns}
		\column{.3\textwidth}
			\includegraphics[width =
			.8\textwidth]{Heatmap_simple.pdf}
			\vspace{.1in}
		\column{.7\textwidth}
%			\footnotesize \color{darkred}Don't overlook
%			heterogeneity in lamb outcomes
	\end{columns}
			\includegraphics[width =
	.2\textwidth]{Heatmap_simple.pdf}
	\end{itemize}
\end{frame}
}

{\usebackgroundtemplate{\transparent{.35}\parbox[c][\paperheight][c]{\paperwidth}{\centering\includegraphics[height=\paperheight]{CoverEwe.jpg}}}
\begin{frame}[t]
	\begin{itemize}
		\item[] \color{navy} \footnotesize Do bighorn sheep populations vary in their social
			structuring?
			\color{darkred} Yes.
	\begin{columns}
		\column{.3\textwidth}
			\includegraphics[width =
			\textwidth]{KeyCompos_simple.pdf}
			\vspace{.1in}
		\column{.7\textwidth}
			\footnotesize \color{darkred} Appropriate management
			strategy may vary from population to population
	\end{columns}
		\item[] \color{navy} \footnotesize Does social structuring account for variation in
			summer lamb survival?
			\color{darkred} Yes.
	\begin{columns}
		\column{.3\textwidth}
			\includegraphics[width =
			\textwidth]{REhists_simple.pdf}
			\vspace{.1in}
		\column{.7\textwidth}
			\footnotesize \color{darkred}Management efforts should
			account for local structuring
	\end{columns}

			\vspace{.1in}
		\item[] \color{navy} \footnotesize Can we leverage
				information on social groups to understand 
			epidemiological rates?
			\color{darkred} Yes.

	\begin{columns}
		\column{.3\textwidth}
			\includegraphics[width =
			.8\textwidth]{Heatmap_simple.pdf}
			\vspace{.1in}
		\column{.7\textwidth}
			\footnotesize \color{darkred}\textbf{Don't overlook
			heterogeneity in lamb outcomes}
	\end{columns}
			\includegraphics[width =
	.2\textwidth]{Heatmap_simple.pdf}
	\end{itemize}
\end{frame}
}
%
%\begin{frame}[t]
%	\frametitle{\color{darkred} Ramifications}
%\vspace{.1in}
%%
%%{\color{navy} Empirical Data}
%%	\begin{itemize}
%%		\item[] Evidence that social
%%			substructuring translates to differential lamb
%%			mortalities
%%	\end{itemize}
%%\vspace{.1in}
%%
%%{\color{navy} Simulations}
%%	\begin{itemize}
%%		\item[] Model is sensitive to lamb mortality
%%			rates
%%%			\begin{itemize}
%%%				\item Suggests research should focus on
%%%					lamb survival post-exposure, as opposed
%%%					to factors driving transmission
%%%			\end{itemize}
%%%		\item Lambs appear to have some
%%%			resistance to pathogens 
%%	\end{itemize}
%%
%%\vspace{.1in}
%%
%%{\color{navy} Ramifications}
%	\begin{itemize}
%		\item[] Direct resources toward understanding differential
%			infection outcomes
%		\item[] Experiment with culling, etc. at subpopulation-level
%	\end{itemize}
%\end{frame}
%


{\usebackgroundtemplate{\transparent{.5}\parbox[c][\paperheight][c]{\paperwidth}{\centering\includegraphics[height=\paperheight]{431WithLamb.jpg}}}

\begin{frame}[t]
	\frametitle{\color{darkred}Thanks!}
\vspace{1.75in}

\begin{columns}[t]
	\column{.5\textwidth}
\scriptsize{
\textcolor{navy}{\textit{Financial support:}}

\begin{itemize}
	\item[] \color{darkred}Penn State Academic Computing Fellowship program
	\item[] \color{darkred}Penn State University Graduate Fellowship program
	\item[] \color{darkred}Morris Animal Foundation
	\item[] \color{darkred}IDFG, WDFW, ODFW, Hells Canyon Bighorn Sheep Restoration Panel 
\end{itemize}
%\vspace{.2in}
}
\column{.5\textwidth}
\scriptsize{
\textcolor{navy}{\textit{Advisor, mentors, collaborators}}
\begin{itemize}
	\item[] \color{darkred}Peter Hudson (PSU)
	\item[] \color{darkred}Frances Cassirer (IDFG)
	\item[] \color{darkred}Paul Cross (USGS)
	\item[] \color{darkred}Raina Plowright (PSU)
	\item[] \color{darkred}Tom Besser (WSU)
	\item[] \color{darkred}Andy Dobson (Princeton)
\end{itemize}
\vspace{.1in}
}
\end{columns}
\end{frame}
}

{\usebackgroundtemplate{\transparent{.5}\parbox[c][\paperheight][c]{\paperwidth}{\centering\includegraphics[height=\paperheight]{431WithLamb.jpg}}}

\begin{frame}[t]
\frametitle{References}
\begin{itemize}
\tiny{

\item[] Besser, T.E., E.F. Cassirer, M.A. Highland, P. Wolff, A. Pierce-Allen,
	K. Mansfield, M.A. Davis, W. Foreyt (2013). ``Bighorn sheep pneumonia:
	Sorting out the cause of a polymicrobial disease''.  \textit{Preventitive
	Veterinary Medicine} \textbf{102}: 85-93.
	\vspace{.1in}

\item[] Plowright, R.K.,\textbf{ K.R. Manlove}, E.F. Cassirer, P.C.Cross, T.
Besser, P.J. Hudson (2013). ``Use of exposure history to identify patterns of
immunity to pneumonia in Bighorn Sheep (\textit{Ovis canadensis})''.
\textit{PLoS One}, \textbf{8}(4): e61919. 

  \vspace{.1in}
   
\item[] Cassirer, E.F., R.K. Plowright, \textbf{K.R. Manlove}, P.C. Cross, A.
Dobson, K. Potter, P.J. Hudson.  ``Spatio-temporal dynamics of pneumonia in
bighorn sheep (\textit{Ovis canadensis}).'' \textit{Journal of Animal Ecology},
\textbf{82}: 518-528.
  \vspace{.1in}
  
\item[] Dassanayake, R. et al. (2010). ``\textit{Mycoplasma ovipneumoniae} can
predispose bighorn sheep to fatal \textit{Mannheimia haemolytica} pneumonia''
Veterinary Microbiology 145: 354-359.
'}
  \end{itemize}
\end{frame}
}

\begin{frame}
	\includegraphics[width = \textwidth]{Besser_2013_PercMoviByAge.pdf}
\end{frame}

\end{document}
